\documentclass[11pt]{article}

\usepackage{kotex}

\begin{document}
Hello world!

\begin{tabular}{r@{.}l}
    3&14159 \\
   16&2 \\
12345&689 \\
\end{tabular}

\begin{tabular}{|l|l|l|} \hline
첫째 줄은 & 높이는 & 기본 높이 \\ \hline
이지만 & 둘째 줄의 & 높이는 \\[1cm] \hline
1cm가 & 추가 & 되었다. \\ \hline
\end{tabular}

\begin{table}[t]
\caption{\texttt{tabular} 환경의 이용 보기\label{tab:tabular}}
\begin{center}
\begin{tabular}{l|cr}
성명 & 주소지 & 월수입 \\ \hline
김철수 & 서울 & 1,000,000 \\
이보라미 & 충청남도 & 600,000
\end{tabular}
\end{center}
\end{table}

\begin{table}[b]
\caption{tabular 환경으로 만든 보기--2\label{tab:tabular2}}
\begin{center}
\begin{tabular}{|r||r@{--}l|*{2}{c|}p{2in}|} \hline
\multicolumn{6}{|c|}{S 사의 주가} \\ \hline \hline
& \multicolumn{2}{|c|}{주가} & & & \\ \cline{2-3}
연도 & \multicolumn{1}{r@{\vline}}{최저} & 
최고 & 거래량 & 액수 & \multicolumn{1}{c|}{특기 사항} \\ \hline
1972 & 97 & 172 & 111 & 12 & 상장 초기로서 특이 사항은 없으나 관심을 받았음 \\ \hline
83 & 130 & 200 & 100 & 900 & 80년대 초 \\ \hline
95 & \multicolumn{5}{|c|}{이 자료는 가상임} \\ \hline
\end{tabular}
\end{center}
\end{table}



\end{document}