\documentclass[11pt]{article}

\usepackage{kotex}
\usepackage{multirow}
\usepackage{colortbl}
\usepackage[table]{xcolor}


\begin{document}
Hello world!

\begin{tabular}{r@{.}l}
    3&14159 \\
   16&2 \\
12345&689 \\
\end{tabular}

\begin{tabular}{|l|l|l|} \hline
첫째 줄은 & 높이는 & 기본 높이 \\ \hline
이지만 & 둘째 줄의 & 높이는 \\[1cm] \hline
1cm가 & 추가 & 되었다. \\ \hline
\end{tabular}

\begin{table}[t]
\caption{\texttt{tabular} 환경의 이용 보기\label{tab:tabular}}
\begin{center}
\begin{tabular}{l|cr}
성명 & 주소지 & 월수입 \\ \hline
김철수 & 서울 & 1,000,000 \\
이보라미 & 충청남도 & 600,000
\end{tabular}
\end{center}
\end{table}

\begin{table}[b]
\caption{tabular 환경으로 만든 보기--2\label{tab:tabular2}}
\begin{center}
\begin{tabular}{|r||r@{--}l|*{2}{c|}p{2in}|} \hline
\multicolumn{6}{|c|}{S 사의 주가} \\ \hline \hline
& \multicolumn{2}{|c|}{주가} & & & \\ \cline{2-3}
연도 & \multicolumn{1}{r@{\vline}}{최저} & 
최고 & 거래량 & 액수 & \multicolumn{1}{c|}{특기 사항} \\ \hline
1972 & 97 & 172 & 111 & 12 & 상장 초기로서 특이 사항은 없으나 관심을 받았음 \\ \hline
83 & 130 & 200 & 100 & 900 & 80년대 초 \\ \hline
95 & \multicolumn{5}{|c|}{이 자료는 가상임} \\ \hline
\end{tabular}
\end{center}
\end{table}

\begin{tabular}{l|l}
제1열 & 제2열 \\ \hline column 1 & column 2
\end{tabular}

{\tabcolsep=0in
\begin{tabular}{l|l}
제1열 & 제2열 \\ \hline column 1 & column 2
\end{tabular}}

\begin{tabbing}
김 철수 \= 통계학과 \\
이 미숙 \> 수학과 \\
야수 오모리 \> 경제학과
\end{tabbing}

\begin{tabbing}
탭 설정만 \= \kill
여기에 \> 탭 설정 \= 새 탭 설정 \+\+ \\
세 번쨰 줄에 출력 \\
\< \< 제1열 \> 제2열 \> 제3열 \\
오른쪽으로 \' 정렬 \- \\
2열 \> 3열 \- \\
1열 \\
\pushtabs
갑 \= 을 \\
 \> 병 \' 계는 오른쪽 끝 \\
\poptabs
가 \> 나 \> 다 \\
\a ={A} \> \a'{A} \> \a'{A}
\end{tabbing}

\begin{tabular}{|l|l|} \hline
전체 행	& 개별 행 \\ \hline
\multirow{4}{*}{네 개의 행} & 1번 행 \\
& 2번 행 \\
& 3번 행 \\
& 4번 행 \\ \hline
합 & 네 개의 행 \\ \hline
\end{tabular}

\begin{tabular}{|>{\columncolor[rgb]{0,.8,0}[0pt]}l|>{\color{white}\columncolor[gray]{.2}[0pt]}l|}
Green & 바탕색 \\
By & \texttt{colorbl} 패키지 \\
And & 회색도 사용
\end{tabular}

\begin{tabular}{|>{\columncolor[rgb]{0,.8,0}}l|>{\color{white}\columncolor[gray]{.2}}l|}
Green & 바탕색 \\
By & \texttt{colorbl} 패키지 \\
And & 회색도 사용
\end{tabular}

\begin{tabular}{|>{\columncolor[rgb]{0,.8,0}[.5\tabcolsep]}l|>{\color{white}\columncolor[gray]{.2}[.5\tabcolsep]}l|}
Green & 바탕색 \\
By & \texttt{colorbl} 패키지 \\
And & 회색도 사용
\end{tabular}

\begin{tabular}{|>{\columncolor[rgb]{0,.8,0}}l|>{\color{white} \columncolor[gray]{.2}}l|}
Green & \cellcolor[gray]{.4}바탕색 \\
By & \texttt{colortbl} 패키지 \\
\rowcolor[rgb]{1,0,1} And & 회색도 사용
\end{tabular}

\setlength\arrayrulewidth{2pt}
\setlength\doublerulesep{2pt}
\arrayrulecolor{blue} \doublerulesepcolor{yellow}
\begin{tabular}{||l||c||} \hline\hline
줄의 & 굵기와 \\ \hline
폭을 & 일부러 \\ \hline
크게 & 하였음 \\ \hline \hline
\end{tabular}

{\rowcolors[\hline]{1}{lime}{}
\begin{tabular}{ll}
\number\rownum & 가나 \\
\number\rownum & 가나 \\
3 & 가나 \\
4 & 가나 \\
\hiderowcolors
5 & 가나 \\
6 & 가나 \\
\end{tabular}
}


\end{document}