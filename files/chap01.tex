\documentclass[11pt]{article}
%\usepackage[hangul]{kotex}  % 각종 기본설정하는 부분 [preamble]
\usepackage[hangul]{kotex}  % 각종 기본설정하는 부분 [preamble]
\usepackage{graphicx}
\title{나의 첫 문서}
\author{김성빈\thanks{인하대학교}}

\begin{document}
\maketitle
\tableofcontents
\section{첫 문서}
나의 첫 \LaTeX\ 문서를 작성하였습니다. 첫 수식이 $y=f(x)=ax^2-bx-c$ 입니다.

두 번째 수식은 전개된(display math) 입니다.
\begin{equation}
f(x) = \frac{1}{2\pi} \exp{-\frac{(x-\mu)^2}{2\sigma^2}}, \ \ \ \ \ \  -\infty < x < \infty
\end{equation}
% 문서의 내용
문서의 내용을 추가 입력합니다. 100\%!

%추가합니다.
위의 식에서 $\mu$는 모평균, $\sigma^2$은 모분산이며 이 함수는 정규분포 $N(\mu, \sigma^2)$의 학률밀도함수입니다. 만일 $\mu=1$, $\sigma=1$이면 표준정규푼포라 하고 함수는
\begin{equation}
f(x)=\frac{1}{2\pi} \exp{-\frac{x^2}{2}}, \ \ \ -\infty < x < \infty
\end{equation}
가 됩니다. 그림은
\begin{center}
\includegraphics[width=3em]{../images/play.png}
\end{center}
로 삽입합니다.

\section{기능 추가}
\LaTeX\ 에서는 사용자가 입력의 위치를 잡기보다는 \LaTeX 의 기능을 사용하는 것이 좋습니다.
\begin{enumerate}
\item 이와 같이 \texttt{enumerate} 환경을 사용하거나,
\item \verb|\section| 명령을 사용하거나
\item \verb|\tableofcontents| 명령을 사용하는 등
\end{enumerate}
의 기능입니다.
\end{document}