\documentclass[11pt]{article}
\usepackage[hangul]{kotex} % 각종 기본설정하는 부분 [preamble]
\usepackage{graphicx}
\usepackage{lipsum}
\title{나의 첫 문서}
\author{김성빈\thanks{인하대학교}}
\begin{document}
\maketitle
\tableofcontents
%\section{lipsum}
%\lipsum
\section{첫 문서\label{ss:normal}}
나의 첫 \LaTeX\ 문서를 작성하였습니다. 첫 
수식이 $y=f(x)=a^2 - bx - c$입니다.
두 번째 수식은 전개된(display math) 입니다.
\begin{equation} \label{eq:eq1}
f(x) = \frac{1}{2\pi} \exp{-\frac{(x-\mu)^2}{2\sigma^2}}, 
\ \ \ -\infty < x < \infty
\end{equation}
% 문서의 내용
문서의 내용을 추가 입력합니다.
% 추가합니다.
위의 식에서 $\mu$는 모평균, $\sigma^2$은 모분산이며 이 함수는 
정규뷴포 $N(\mu, \sigma^2)$의 확률밀도함수입니다. 만일 $\mu=1$, 
$\sigma=1$이면 표준정규뷴포라 하고 함수는
\begin{equation} \label{eq:eq2}
f(x) = \frac{1}{2\pi} \exp{-\frac{x^2}{2}}, 
\ \ \ -\infty < x < \infty
\end{equation}
가 됩니다. 그림은
\begin{figure}
\begin{center}
\includegraphics[width=3em]{../images/play.png}
\end{center}
\caption{그림 제목 들어가요.} \label{fig:play2}
\end{figure}
로 삽입합니다.
\section{기능 추가 \label{ss:add}}
\LaTeX 에서는 사용자가 입력의 위치를 잡기보다는 \LaTeX 의 
기능을 사용하는 것이 좋습니다.
\begin{enumerate}
\item 이와 같이 \texttt{enumerate} 환경을 사용하거나,
\item \verb|\section| 명령을 사용하거나
\item \verb|\tableofcontents| 명령을 사용하는 등
\end{enumerate}
의 기능입니다.
\section{인용해 보기}
제 \ref{ss:normal}절의 식 (\ref{eq:eq1})\은 정규분포일 때의
확률밀도함수이며 그림 \ref{fig:play2}\와는 아무 관계가 없다.
\pageref{eq:eq1}면의 식 (\ref{eq:eq1})의 특수한 경우로 $\mu=0$,
$\sigma=1$이면 식 (\ref{eq:eq2})가 된다. 제\ref{ss:add}절은
\pageref{ss:add}면에 나온다.
\end{document}














\end{document}