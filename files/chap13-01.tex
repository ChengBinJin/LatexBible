\documentclass{beamer}
%\usepackage{syshan}
\usepackage{kotex}
\usepackage{colortbl}

\usetheme{default}
\usecolortheme{beaver}
\usefonttheme{professionalfonts}
\useinnertheme{circles}
\useoutertheme{infolines}

\title{중심극한정리}
\subtitle{$n$이 클 때의 $\bar{X}$의 분포}
\author[심송용]{\texttt{sysim@hallym.ac.kr}}
\date[\today]{한국통계학회 2007년 추계학술대회}
\institute{한림대 통계학과}

\begin{document}
\setbeamercovered{transparent=25}

\begin{frame}
\titlepage
\end{frame}

\begin{frame}
\frametitle{Central Limit Theorem}
\begin{theorem}<1->
$X_1, X_2, \cdots, X_n$이 독립이고 $E[X] = \mu$, $Var(X)=\sigma^2 (>0)$ 이고 $\bar{X}$와 $S^2$이 각각 표본평균, 표본분산이라고 하자. 이때 $n \rightarrow \infty$ 이면 
\begin{equation} \label{1}
\frac{\bar{X} - \mu}{S/ \sqrt{n}} \stackrel{asymp.}{\sim}N(0,1)
\end{equation}
이다.
\end{theorem}

\begin{proof}[Sketch of proof]<2->
\begin{enumerate}
\item<3-> 식 (\ref{1})의 $ch.f$ $\phi(t)$의 expansion을 구한다.
\item<4-> $n \rightarrow \infty$일 때 이 함수가 수렴함을 보인다.
\end{enumerate}
\end{proof}
\end{frame}

\begin{frame}
\begin{columns}[T]
\begin{column}{3in}
\centerline{
\includegraphics[width=2in, height=3in]{../images/mari.png}}
\end{column}
\begin{column}{8cm}<2->
\begin{itemize}
\item<1-> 통계의 미술
\item<2-> 5\% 의 진정한 의미
\item<3> 등등
\end{itemize}
\end{column}
\end{columns}
\end{frame}

\begin{frame}
\frametitle{Row-wise display}
\centerline{
\begin{tabular}{l!{\vrule}rrrrr}
\rowcolor[rgb]{.9, 0, 0} age group & 10s & 20s & 30s & 40s & 합 \\ \hline
\rowcolor[gray]{.7} \pause 남 & 10 & 5 & 78 & 45 & 138 \\
\rowcolor[rgb]{0, .9, 0} \pause 여 & 15 & 34 & 2 & 12 & 63 \\
\rowcolor[rgb]{0, 0, .9} \pause 합 & 25 & 39 & 80 & 57 & 201 \\ \hline
\end{tabular}}
\end{frame}

\begin{frame}
\frametitle{Row-wise display}
\begin{tabular}{l| r<{\onslide<2->} r<{\onslide<3->} r<{\onslide<4->} r<{\onslide<5>} r} \hline
age group & 10s & 20s & 30s & 40s & 합 \\ \hline
남 & 10 & 5 & 78 & 45 & 138 \\
여 & 15 & 34 & 2 & 12 & 63 \\
합 & 25 & 39 & 80 & 57 & 201 \\ \hline
\end{tabular}
\end{frame}

\end{document}