\documentclass{beamer}
%\usepackage{syshan}
\usepackage{kotex}
\usepackage{colortbl}
\usepackage{tikz}
\usepackage{hyperref}
\usepackage{multimedia}
\usepackage{xcolor}

\usetheme{default}
\usecolortheme{beaver}
\usefonttheme{professionalfonts}
\useinnertheme{circles}
\useoutertheme{infolines}

\title{중심극한정리}
\subtitle{$n$이 클 때의 $\bar{X}$의 분포}
\author[심송용]{\texttt{sysim@hallym.ac.kr}}
\date[\today]{한국통계학회 2007년 추계학술대회}
\institute{한림대 통계학과}

\renewcommand{\insertgotosymbol}{$\Uparrow$}
\renewcommand{\insertskipsymbol}{$\Rightarrow$}
\renewcommand{\insertreturnsymbol}{$\Leftarrow$}

\setbeamercolor{block title}{bg=green}
\setbeamercolor{frametitle}{fg=brown, bg=brown!20}
%\beamersetaveragebackground{yellow}
\beamertemplateshadingbackground{blue!20}{blue!80}

\begin{document}
\setbeamercovered{transparent=25}

\begin{frame}
\titlepage
\end{frame}

\begin{frame}
\frametitle{Central Limit Theorem}
\begin{theorem}<1->
$X_1, X_2, \cdots, X_n$이 독립이고 $E[X] = \mu$, $Var(X)=\sigma^2 (>0)$ 이고 $\bar{X}$와 $S^2$이 각각 표본평균, 표본분산이라고 하자. 이때 $n \rightarrow \infty$ 이면 
\begin{equation} \label{1}
\frac{\bar{X} - \mu}{S/ \sqrt{n}} \stackrel{asymp.}{\sim}N(0,1)
\end{equation}
이다.
\end{theorem}

\begin{proof}[Sketch of proof]<2->
\begin{enumerate}
\item<3-> 식 (\ref{1})의 $ch.f$ $\phi(t)$의 expansion을 구한다.
\item<4-> $n \rightarrow \infty$일 때 이 함수가 수렴함을 보인다.
\end{enumerate}
\end{proof}
\end{frame}

\begin{frame}
\begin{columns}[T]
\begin{column}{3in}
\centerline{
\includegraphics[width=2in, height=3in]{../images/mari.png}}
\end{column}
\begin{column}{8cm}<2->
\begin{itemize}
\item<1-> 통계의 미술
\item<2-> 5\% 의 진정한 의미
\item<3> 등등
\end{itemize}
\end{column}
\end{columns}
\end{frame}

\begin{frame}
\frametitle{Row-wise display}
\centerline{
\begin{tabular}{l!{\vrule}rrrrr}
\rowcolor[rgb]{.9, 0, 0} age group & 10s & 20s & 30s & 40s & 합 \\ \hline
\rowcolor[gray]{.7} \pause 남 & 10 & 5 & 78 & 45 & 138 \\
\rowcolor[rgb]{0, .9, 0} \pause 여 & 15 & 34 & 2 & 12 & 63 \\
\rowcolor[rgb]{0, 0, .9} \pause 합 & 25 & 39 & 80 & 57 & 201 \\ \hline
\end{tabular}}
\end{frame}

\begin{frame}
\frametitle{Row-wise display}
\begin{tabular}{l| r<{\onslide<2->} r<{\onslide<3->} r<{\onslide<4->} r<{\onslide<5>} r} \hline
age group & 10s & 20s & 30s & 40s & 합 \\ \hline
남 & 10 & 5 & 78 & 45 & 138 \\
여 & 15 & 34 & 2 & 12 & 63 \\
합 & 25 & 39 & 80 & 57 & 201 \\ \hline
\end{tabular}
\end{frame}

\begin{frame}
\frametitle{\texttt{tikz} 패키지}
\begin{center}
\begin{tikzpicture}[yscale=5, xscale=1]
\node<1->[red] at (1, 0.6) {정규분포 곡선};
node[below=0.8cm, white] at (1,0)
{$f(x) = \frac{1}{\sqrt{2\pi}} \Big\{-\frac{(x-\mu)^2}{2\sigma^2} \Big\}$};
\draw<1->[thick, ->] (-3.5, 0) -- (5.5, 0);
\draw<1->[black, domain=-3:3] plot(\x, {1/sqrt{2*pi}*exp(-pow(\x,2)/2})
node[below] at (0,0) {\small $\mu=0$};
\draw<1->[dotted] (0,0) -- (0,0.3989423); %평균선
%% 두 번째 슬라이드
\onslide<2->{
\draw[red, domain=-2:4]
plot(\x, {1/sqrt(2*pi)*exp(-pow(\x-1,2)/2)})
node[below] at (1,0) {\small $\mu=1$};
\draw[red, dotted] (1,0) -- (1,0.3989423); %평균선
}
%% 세 번째 들라이드
\only<3->{
\draw[blue, domain=-1:5]
plot(\x, {1/sqrt(2*pi)*exp(-pow(\x-2,2)/2)})
node[below] at (2,0) {\small $\mu=2$};
\draw[blue, dotted] (2,0) -- (2,0.3989423); %평균선
}
%% 마지막 슬라이드
\draw<4> node[below=0.8cm] at (1,0)
{$f(x) = \frac{1}{\sqrt{2\pi}}
\Big\{-\frac{(x-\mu)^2}{2\sigma^2}\Big\}$};
\end{tikzpicture}
\end{center}
\end{frame}

\begin{frame}
\href{http://jupiter.hallym.ac.kr}{홈페이지 보기}
\end{frame}

\begin{frame}
\begin{itemize}[<+->]
\item 통계의 미술
\item 5\% 의 진정한 의미
\item 등등
\end{itemize}
\hypertarget<2>{item:two}{}
\end{frame}

\begin{frame}
\hyperlink{item:two}{Jump to the 2nd item}

\hyperlink{item:two}{\beamerbutton{Jump to the 2nd item}}
\end{frame}

\begin{frame}
\hyperlink{page:table1}{\beamerbutton{Go to table 1}}
\hyperlink{page:first}{\beamerreturnbutton{Jump to the first page}}
\hyperlink{item:two}{\beamergotobutton{Jump to the 2nd item}}
\hyperlink{page:last}{\beamerskipbutton{Jump to the last page}}
\hyperlinkslideprev{\beamerskipbutton{prev}}
\hyperlinkslidenext{\beamerskipbutton{next}}
\end{frame}

%\begin{frame}
%\frametitle{Movie Embeding Effects}
%\begin{center}
%\begin{tabular}{|c|c|c|} \hline
%Movie 1 & Movie 2 & Movie 3 \\ \hline
%\movie[boardwidth=1pt, loop, showcontrols=true, label=Movie1, %poster, width=100pt, height=80pt]{}{./multimedia/twing.avi} &
%\movie[borderwidth=0pt, loop, start=0.5s, showcontrols=true, label=Movie2, width=3cm, height=2.4cm]{Twing.avi}{./multimedia/twing.avi} &
%\movie[borderwidth=0pt, loop, label=Movie3, externalviewer]{\underline{Play Movie}}{./multimedia/twing.avi} \\ \hline
%\end{tabular}
%\end{center}
%\end{frame}

\begin{frame}
%\hyperlinkmovie[resume]{Movie1}{\beamerbutton{Resume Movie 1}}
%\hyperlinkmovie[stop]{Movie2}{\beamerbutton{Stop Movie 1}}
%\hyperlinkmovie[pause]{Movie2}{\beamerbutton{Pause Movie 2}}
%\hyperlinkmovie[resume]{Movie2}{\beamerbutton{Resume Movie 2}}
\end{frame}

\end{document}