\documentclass[11pt]{article}

\usepackage{kotex}
%\usepackage{ulem}

\hyphenation{Ghost-Script}
\showhyphens{difficult}

\font \myfont=manfnt scaled 1095

\begin{document}
%\LaTeX\    은 문장 안에서 하나     이상의 공간은   한 개의 공간으로 취급한다.

%\setlength{\parindent}{1cm}
%\LaTeX\ 의    문단은 한 줄 이상의 빈 줄을 주는 것으로 만들어진다.

%여기서 \newline\newline 한 줄 \\ Here go\newline \null \newline two empty lines.

%여기서 \newline 한 줄 \\ Here go \\ \\ two empty lines

%여기에 한 줄 \\[2cm]
%Here go \\ an empty lines

%또, 따른 한 줄을\linebreak 중간에서 바꾸는 방법

%`작은따움표'와 ``큰따옴표''는 만들기 쉽다.

%``\,`큰' 따옴표는 `크다'.\,''

%-, --, ---

%텍스트로 그림을 저장한 Po\-st\-Script 파일을 화면으로 볼 수 있게 하는 프로그램으로 Gho\-st\-Script가 있다.

%텍스트로 그림을 저장한 PostScript 파일을 화면으로 볼 수 있게 하는 프로그램으로 Ghostscript가 있다.

%여기의 Ghostscript는 Ghost-Script로 하이픈을 무지하게 하고 싶은 Ghostscript 이다. 이 출력은 \verb|\hyphenation| 선언이 위와 같이 주어진 경우의 출력이다. \\
We have I + I = II. Right? \\

We have I + I = II\@.  Right? \\
Johnson et al. ABC \\
Johnson et al.\ ABC

\# \$ \% \& \_ \{ \} \verb|~| \verb|^| \verb|\| \symbol{64}

{\myfont \symbol{40} \symbol{41} \symbol{42} \symbol{43}%
\symbol{44} \symbol{45} \symbol{46} \symbol{42}} \space \space
{\myfont \symbol{47} \symbol{48} \symbol{49} \symbol{50} % 
\symbol{51} \symbol{52} \symbol{53} \symbol{49}}

{\myfont \char66}{\myfont \char67}{\myfont \char68}%
{\myfont \char71}{\myfont \char72}{\myfont \char73}%
{\myfont \char74}{\myfont \char68}%

오늘은 \today 이다.

\TeX $\rightarrow$ \LaTeX $\rightarrow$ \LaTeXe

여기에 \underline{밑줄} 쫙

\underline{밑줄이 길면 한 줄을 넘어가며 줄 바꾸기가 안됩니다.}

이 부분 \emph{중요}합니다.

This is \emph{emphasized} text.

\emph{did}not, \emph{did\/}not \\ \\

\dotemph{드러냄표}로 강조하기

\LaTeX 에서는 \% 이후 % 여기서 입력은 무시한다. % 아무거나 $ &

$\mbox{\L}(x) = \mbox{\AE} + \mbox{\^\i}$

각 주의 번호가 증가하므로 \\
\verb+\setcounter+\footnotemark 나 \\ 
\verb+\addtocounter+\addtocounter{footnote}{-1}\footnote{\ref{ss:cntnum}절 참조} 명령을 이요하여...
\end{document}