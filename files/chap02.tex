\documentclass[11pt]{article}

\usepackage{kotex}

\hyphenation{Ghost-Script}

\begin{document}
This space that space                  This space that    space
\LaTeX \       은 문장 안에서 하나   이상의 공간은
한 개의 공간으로 취급한다.

\LaTeX 의    문단은 한 줄 
이상의 빈 줄을 주는
것으로 만들어진다.

\setlength{\parindent}{5cm} \LaTeX 의    문단은 한 줄 
이상의 빈 줄을 주는
것으로 만들어진다.

\setlength{\parindent}{0cm}\LaTeX \ 의문단은 한 줄 이상의 빈 줄을 주는 것으로 만들어진다.

\LaTeX \ 의문단은 한 줄 이상의 빈 줄을 주는 것으로 만들어진다.

\setlength{\parindent}{5cm}
\LaTeX \ 의문단은 한 줄 이상의 빈 줄을 주는 것으로 만들어진다.

\LaTeX \ 의문단은 한 줄 이상의 빈 줄을 주는 것으로 만들어진다.

\noindent
\LaTeX \ 의문단은 한 줄 이상의 빈 줄을 주는 것으로 만들어진다.

\parskip=2cm
\parindent=1cm
\LaTeX \ 의문단은 한 줄 이상의 빈 줄을 주는 것으로 만들어진다.

\LaTeX \ 의문단은 한 줄 이상의 빈 줄을 주는 것으로 만들어진다.

여기에 \newline \newline 한 줄\\
Here go \newline \null
\newline two empty lines

여기에 \newline 한 줄 \\
Here go \\ \\
two empty lines

여기에 한 줄 \\[2cm]
Here go \\ an empty lines

또, 다른 한 줄을 \linebreak 중간에서 바꾸는 방법

\newpage \null
\newpage

Hello world!

\parskip=0.5cm

`작은따옴표'와 ``큰따옴표"는 만들기 쉽다.

``\,`큰' 따옴표는 `크다'.\,"

-, --, ---

텍스트로 그림을 저장한
Po\-st\-Script 파일을 화면으로
볼 수 있게 하는 프로그램으로
Gho\-st\-Script가 있다.

텍스트로 그림을 저장한 PostScript 파일을 화면으로 볼 수 있게 하는 프로그램으로  Ghostscript가 있다.

We have I + I = II. Right?

We have I + I = II\@. Right?

Johnson et al. ABC

Johnson et al. \ ABC

\# \$ \% \_ \{ \} \~ \^

\font \myfont=manfnt scaled 1095
{\myfont \symbol{40} \symbol{41} \symbol{42} \symbol{43}%
 \symbol{44} \symbol{45} \symbol{46} \symbol{42}} \space \space
 
{\myfont \symbol{47} \symbol{48} \symbol{49} \symbol{50}%
\symbol{51} \symbol{52} \symbol{53} \symbol{49}}

{\myfont \char66}{\myfont \char67}{\myfont \char68}%
{\myfont \char71}{\myfont \char72}{\myfont \char73}%
{\myfont \char74}{\myfont \char68}

오늘은 \today 이다.

\TeX $\rightarrow$ \LaTeX $\rightarrow$ \LaTeXe

여기에 \underline{밑줄} 쫙. \underline{밑줄이 길면 한줄을 넘어가며 줄 바꾸기가 안됩니다.}

이 부분 \emph{중요} 합니다.

This is \emph{emphasized} text.

\emph{did} not, \emph{did\/} not

\dotemph{드러냄표}로 강조하기

\LaTeX 에서는 \% 이후 % 여기서
입력은 무시한다. % 아무거나 $ &

\`{o} \'{o} \^{o} 

\"{o} \~{o} \={o}

\.{o} \u{o} \v{o}

\H{o} \t{oo} \c{o}

\b{o} \d{o} \r{o}

\oe \  \OE \  \ae \  \AE

\aa \AA \o \O

\l \L \ss ?`

!`

$\mbox{\L}(x) = \mbox{\AE} + \mbox{\^\i}$ \footnote{\LaTeX에서 [ 와 ] 사이에 들어가는 내용은 옵션이므로 항상 생략할 수 있다.}

각 주의 번호가 증가하므로 \\
\verb+\setcounter+\footnotemark 나 \\
\verb+\addtocounter+\addtocounter{footnote}{-1}\footnote{\ref{ss:cntnum}절 참조}
명령을 이용하여... \footnote{이 경우 \texttt{\$}는 \emph{robust}이기 때문에 굳이 \texttt{\symbol{92}protect}를 사용할 필요가 없다.}

\section{ \( a+b \) }

\section{ \protect\( a+b \protect\)}

\end{document}























