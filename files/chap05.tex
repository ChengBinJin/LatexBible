\documentclass[11pt]{article}

\usepackage{kotex}
\usepackage{bm}
\usepackage{amsmath, amsfonts}
\usepackage{subeqnarray}
\usepackage{dotseqn}

\newcommand{\sumnbr}{\sum_{\mbox{nbr}}}
\newcommand{\infsup}{\mathop{\rm inf\,sup}}

\begin{document}
$y=2x-1$ \\

\(y=2x-1\)

\begin{math}
y=2x-1
\end{math}

$x^2-1$

$$x^2-1$$

$x^{2y}, x^{{2y}^x}, X_{n_1}^{2y^z}$

$2^{2^{2^{2^{2^{2^{2^{2}}}}}}}$

$f'(x) \quad f'''(x)|_{x=0}$
 
$\pi, \Phi, \Sigma, \mu, \alpha$

${\mit \Gamma \Pi \Phi}$는 $\Gamma \Pi \Phi$와 다르다.

$\mathit{\Psi \Theta \Omega}$는 $\Psi \Theta \Omega$와 다르다. 

$\sqrt[n]{x}, \sqrt[3]{ax+b}, \sqrt[2]{5}, \sqrt{2}, \sqrt[x]{2}$

\begin{equation} \label{eq:sqrt}
\sqrt{1+\sqrt{1+\sqrt{1+\sqrt{1+\sqrt{1+\sqrt{1+x}}}}}}
\end{equation}

$$\sqrt{\mathstrut a} \quad \sqrt{\mathstrut d} \quad \sqrt{\mathstrut g}$$

$(x_1+\cdots+x_n)$

$(a_1, \ldots, a_m)$

$\ldots$(...)

$\displaystyle \frac{x^2+1}{y_1^2-1}$

$1+\frac{1}{1+\frac{1}{1+\frac{1}{1+\frac{1}{\frac{1}{1+x}}}}}$

$\frac12$, $\frac x2$

${\cal S}$를 ${\cal S}=\{A \bigm|A \ni{\cal T}\}$라 하자.

$\not0, \emptyset$

$\not\ni, \not\subset, \not<$

$\lim_{n\to\infty}$

$$\lim_{n\to\infty}$$

$\limsup_{n}$

$$\liminf_{n\longrightarrow\infty}$$

$$liminf_{n\longrightarrow\infty}$$

$a \bmod b \qquad y \pmod{a+b}$

$\int\!\!\int \cdots \int f dP$

$1/\!\log n \qquad 1/\log$

$\sqrt{4} \, n$

$f(x;\mu,\sigma) = \frac{1}{\sqrt{2\pi}\sigma} \exp \Bigl\{-\frac{(x-\mu)^2}{2\sigma^2} \Bigr\}$

\[
f(x;\mu,\sigma) = \frac{1}{\sqrt{2\pi}\sigma} \exp \Bigl\{-\frac{(x-\mu)^2}{2\sigma^2} \Bigr\}
\]

\[
f(x;\mu,\sigma) = \frac{1}{\sqrt{2\pi}\sigma} \exp \{-\frac{(x-\mu)^2}{2\sigma^2} \}
\]

디스플레이 스타일:
\[\sum_{i=1}^n x_i = \int_0^1 f \]
텍스트 스타일:
$\sum_{i=1}^n x_i = \int_0^1 f$

$\displaystyle \frac{a-b}{c+d}$ 와 $\frac{a-b}{c+d}$

$\displaystyle{\frac{a-b}{c+d}}$

$\frac{\displaystyle a-b}{\displaystyle c+d}$

$\vec{x} + \vec{y} = \left \{ 
\begin{array}{l} a \\ b \end{array}
\right.$

$\textbf{A} = \left( 
\begin{array}{ccc} 
a_{11}& a_{12} & a_{13} \\
a_{21}& a_{22}& a_{23} \\
a_{31}& a_{32}& a_{33} \\
\end{array} 
\right)$

$$\widehat{a-1} = \widetilde{x-y} + \widehat{\mbox{Cov}}$$

$\begin{array}{lrc}
a & b & c\\
a-b & b-c & c-a \\
x^2+2x+1 & x^2+2x+1 & x^2+2x+1 \\
\end{array}$

\begin{equation}
\begin{array}{*{3}{c@{+}}c@{=}c}
a_{11}x_1 & a_{12}x_2 & \cdots & a_{1n}x_n & b_1 \\
a_{11}x_1 & a_{12}x_2 & \cdots & a_{1n}x_n & b_1 \\
\multicolumn{5}{c}{\vdots} \\
a_{11}x_1 & a_{12}x_2 & \cdots & a_{1n}x_n & b_1 \\
\end{array}
\end{equation}

$\left( 
\begin{array}{c}
\left| 
\begin{array}{cc}
a & b \\
c & d \\
\end{array} 
\right| \\
e \\ 
f \\
\end{array} 
\right)$

$$x^n = \overbrace{x \times x \times \cdots \times x}$$

$$\overbrace{a+\underbrace{b+\overline{c+d}}+e}$$

$$\overbrace{a+\underbrace{b+c}_{123}+e}^{ab}$$

\begin{equation} \label{eq:logodds}
\frac{p(x_i|\bm{x}_{-i})}{1-p(x_i|\bm{x}_{-i})} = 
\theta_1 \sum_{i=1}^m x_i + \beta_1 \sumnbr x_ix_{i'}
\end{equation}

\begin{eqnarray}
(x+y)^2 & = & x^2 + xy + yx + y^2 \label{eqnarray1} \\
		& = & x^2 + xy + xy + y^2 \nonumber \\
		& = & x^2 + 2xy + y^2 \label{eqnarray2}
\end{eqnarray}

\begin{eqnarray}
(x+y)^2 & = & x^2 + xy + yx + y^2 \\
		& = & x^2 + 2xy + y^2 \nonumber
\end{eqnarray}

\begin{eqnarray*}
\lefteqn{a+b+c+d+e+f+g+h+i+j+k+l=}  \\
& & x+y+z+a+b+c+d+e+f+g+o+s+t+ \\
& & u+v+w
\end{eqnarray*}

Math italic $different$ is from \emph{different}.

\begin{equation}
f(x) = \left\{ \begin{array}{rl}
x & if x > 2 or if x < -2 \\
x & {\rm if } x > 2 {\rm or if} x < -2 \\
x & \mbox{if } x > 2 \mbox{ or if } x < -2
\end{array}
\right.
\end{equation}

$\mathrm{Form\ e^{pdf} + \Phi(x)}$ \\
$\mathit{Form\ e^{pdf} + \Phi(x)}$ \\
$\mathtt{Form\ e^{pdf} + \Phi(x)}$ \\
$\mathbf{Form\ e^{pdf} + \Phi(x)}$ \\
$\mathsf{Form\ e^{pdf} + \Phi(x)}$ \\
$\mathcal{ABC}$

$${\bf a} = (a_1, a_2, \ldots, a_n)^T$$

\boldmath{$a = (a_1, a_2, \ldots, a_n)^T$}\unboldmath

\boldmath{$a \mbox{\unboldmath $=(a_1, a_2, \ldots, a_n)^T$} $} \unboldmath

$\bm{a} = (a_1, a_2, \ldots, a_n)^T$

$\pmb{aX + \beta + \gamma}$

$\boldsymbol{aX + \beta + \gamma}$ 

$A\stackrel{f}{\to}B\stackrel{g}{\to}C$

$A \small{A} \LARGE{A} \Huge{A} \footnotesize{A}$

${\displaystyle\frac{a+b}{c-d}}$

${\frac{\textstyle a+b}{\textstyle c-d}}$

$1+\frac{1}{1+\frac{1}{1+\frac{1}{1+\frac{1}{1+\frac{1}{1+x}}}}}$ \\

$1+\displaystyle\frac{1}{1+\displaystyle\frac{1}{1+\displaystyle\frac{1}{1+\displaystyle\frac{1}{1+\displaystyle\frac{1}{1+x}}}}}$

\begin{equation}
\mbox{표준편차} = \sqrt{\mbox{분산}} = \sqrt{\frac{\mbox{편차}^2 \mbox{의 합}}{\mbox{표본의 개수} - 1}} \\
\end{equation}

$\int_0^\infty f(x) dx$

$\int\limits_0^\infty f(x) dx$

${\displaystyle \int\limits_0^\infty f(x) dx}$

$\int\nolimits_0^\infty f(x) dx$

${\displaystyle \int_0^\infty f(x) dx}$

$\infsup_{n\rightarrow \infty} f_n(x)$

${\displaystyle \infsup_{n\rightarrow \infty} f_n(x)}$

$\infsup \limits_{n\rightarrow \infty}^{\rm woops} f_n(x)$

$\infsup \nolimits_{n\rightarrow \infty}^{\rm woops} f_n(x)$

$$\sum_{i, j=1, n \atop i \ne j}$$

${\displaystyle \sum_{i,j=1,n \\ i\ne j}}$

$2n \choose n$, ${}_{2n} \rm C_n$

$\left(\begin{array}{c} 2n \\ n \end{array}\right)$

$$(_{\ n\,}^{2n})$$

$x \brack 2y$

$a-c \brace b$

$${}_{ 73}^{231}{\rm U}$$

$${}_{\phantom{0}73}^{231}{\rm U}$$

$$x+{a+b \atopwithdelims<> c}$$

$$x+{a+b \atopwithdelims\uparrow\downarrow c}$$

%$$A=\pmatrix{\lambda & l \cr a & \alpha \cr}$$

$$B = 
\begin{pmatrix}
\lambda & l \\
a & \alpha \\
\end{pmatrix}$$

$$B=\left(
\begin{array}{cc}
\lambda & l \\
a & \alpha \\
\end{array}
\right)$$

$$A=\bordermatrix{& n_1 & n_2 & n_3 \cr
m_1 & A_{11} & A_{12} & A_{13} \cr
m_2 & A_{21} & A_{22} & A_{23} \cr}$$

\[
A = 
\begin{array}{cccc}
& n_1 & n_2 & n_3 \\
\begin{array}{c} m_1 \\ m_2 \end{array} &
\left(
\begin{array}{c} A_{11} \\ A_{21} \end{array} \right. &
\begin{array}{c} A_{12} \\ A_{22} \end{array} &
\left. \begin{array}{c} A_{13} \\ A_{23} \end{array}\right)
\end{array}
\]

$$f(x)= \left\{
\begin{array}{ll} 
x & \mbox{for } x>0 \\
-x & \mbox{for } -1<x \leq 0 \\
x^2 & \mbox{otherwise} 
\end{array}
\right.$$

$$f(x)= \left\{
\begin{array}{rl} 
x & \mbox{for } x>0 \\
-x & \mbox{for } -1<x \leq 0 \\
x^2 & \mbox{otherwise} 
\end{array}
\right.$$

$${a+b \over c+d}$$ $$\frac{a+b}{c+d}$$ 


$$\{x \Big| x \in X \}$$

$$\{x \Bigm| x\in X \}$$

$$\{x \Bigl| x\in X \}$$

$$\{x \Bigr| x\in X \}$$

$$\Biggl\{ \biggl\{ \Bigl\{ \bigl\{ \{ \} \bigr\} \Bigr\} \biggr\} \Biggr\}$$

$$|b-|x+y||$$

$$\Bigl| b-|x+y| \Bigr|$$

\begin{subeqnarray}
&& \frac{1}{\sqrt{2\pi} \sigma} \exp\{-\frac{(x-\mu)^2}{2\sigma^2}\} \mbox{ if } \mu=0, \sigma=1 \slabel{eq:teqn} \\
&& \frac{1}{\sqrt{2\pi} \sigma} \exp\{-\frac{x^2}{2\sigma^2}\} \mbox{ if } \mu=0 \slabel{eq:mu=0} \\
&& \frac{1}{\sqrt{2\pi}} \exp \{-\frac{x^2}{2}\} \mbox{ if } \mu=0, \sigma=1 \slabel{eq:std}
\end{subeqnarray}
식 \ref{eq:mu=0}\와 \ref{eq:std}\는 식 \ref{eq:teqn}의 특별한 경우이다.

$$\mathbb{A B C D E R}$$

$$\mathfrak{Mathfrack Font}$$

\circledR \checkmark \maltese
\end{document}