\documentclass[11pt]{article}

\usepackage{kotex}
\usepackage{latexsym}
\usepackage{amssymb}
\usepackage{bm, amsfonts, amsmath, amsbsy}

\newcommand{\sumnbr}{\sum_{\mbox{nbr}}}
%\newcommand{\bm}[1]{\mbox{\boldmath{$#1$}}}
\newcommand{\infsup}{\mathop{\rm inf\, sup}}

\begin{document}
Hello Wolrd!

$y=2x-1$ 

\(y=2x-1\)

\begin{math}
y=2x-1
\end{math}

$$y=2x-1$$

$x^{2y}, x^{{2y}^x}, X_{n_1}^{2y^z}$ \\ \\

$X^{2m}_{3m}, X^2n_3m$ \\

$X^2_n, X_n^2, X_{n^2}$ \\

$2^{2^{2^{2^{2^{2^{2^{2}}}}}}}$ \\

$f'(x) \quad f'''(x)|_{x=0}$ \\

$\pi, \Phi, \Sigma, \mu, \alpha$ \\

${\mit \Gamma \Pi \Phi}$는 $\Gamma \Pi \Phi$와 다르다. \\

... $\mathit{\Psi \Theta \Omega}$는 $\Psi \Theta \Omega$와 다르다. \\

$\sqrt[n]{x}, \sqrt[3]{ax+b}, \sqrt[2]{5}, \sqrt{2}, \sqrt[x]
{2}$ \\

\begin{equation} \label{eq:sqrt}
\sqrt{1+\sqrt{1+\sqrt{1+\sqrt{1+\sqrt{1+\sqrt{1+x}}}}}}
\end{equation} \\

$$\sqrt{\mathstrut a} \sqrt{\mathstrut d} \sqrt{\mathstrut g}$$

$(x_1+\cdots+x_n)$

$(a_1, \ldots, a_m)$

$(a_1, \vdots, a_n)$

$(a_1, \ddots, a_n)$

$\displaystyle \frac{x^2+1}{y_1^2-1}$ \\

$\displaystyle 1+\frac{1}{1+\frac{1}{1+\frac{1}{1+\frac{1}{1+\frac{1}{1+x}}}}}$

$\frac12$ $\frac x2$

${\cal A} \otimes$

${\cal A} \unlhd$

${\cal S}$를 ${\cal S}=\{A \bigm| A \ni{\cal T}\}$라 하자.

$\not 1, \emptyset$

$\not\ni, \not\subset, \not<$

$\lim_{n\to\infty}$ \\

$\displaystyle \liminf_{n\longrightarrow \infty}$ \\

$\displaystyle liminf_{n\longrightarrow \infty}$ \\

$a \bmod b \qquad y \pmod{a+b}$

$\int \int \cdots \int f dP$

$\int \! \int \cdots \int f dP$

$\int \!\! \int \cdots \int f dP$

$1/ \! \log n$

$\sqrt{4} \, n$

\[
f(x;\mu,\sigma) = \frac{1}{\sqrt{2\pi}\sigma} \exp\Bigl\{-\frac{(x-\mu)^2}{2\sigma^2} \Bigr\}
\]

$f(x;\mu,\sigma) = \frac{1}{\sqrt{2\pi}\sigma} \exp\Bigl\{-\frac{(x-\mu)^2}{2\sigma^2} \Bigr\}$

$$\sum_{i=1}^n x_i = \int_0^1 f$$

$\sum_{i=1}^n x_i = \int_0^1 f$

$\displaystyle{\frac{a-b}{c+d}}$

$\frac{\displaystyle a-b}{\displaystyle c+d}$

$$\displaystyle \vec{x} + \vec{y} = \left\{\begin{array}{l} a \\ b \end{array}\right.$$

\[
\textbf{A} = \left(\begin{array}{ccc} a_{11}& a_{12}& a_{13}\\
a_{21}& a_{22}& a_{23}\\
a_{31}& a_{32}& a_{33}
\end{array} \right)
\]

$$\widehat{a-1} = \widetilde{x-y} + \widehat{\mbox{Cov}}$$

\[
\begin{array}{lrc}
a & b & c \\
a-b & b-c & c-a \\
x^2+2x+1 & x^2+2x+1 & x^2+2x+1
\end{array}
\]

\begin{equation}
\begin{array}{*{3}{c@{+}}c@{=}c}
a_{11}x_1 & a_{12}x_2 & \cdots & a_{1n}x_n & b_1 \\
a_{11}x_1 & a_{12}x_2 & \cdots & a_{1n}x_n & b_1 \\
\multicolumn{5}{c}{\vdots} \\
a_{11}x_1 & a_{12}x_2 & \cdots & a_{1n}x_n & b_1
\end{array}
\end{equation}

$\left( \begin{array}{c} 
\left| \begin{array}{cc} 
a & b \\
c & d
\end{array} \right| \\
e \\
f 
\end{array} \right) $

$$x^n = \overbrace{x\times x \times \cdots \times x}$$

$$\overbrace{a+\underbrace{b+\overline{c+d}}+e}$$

$$\overbrace{a+\underbrace{b+c}_{123}+e}^{ab}$$

\begin{equation} \label{eq:logodds}
\frac{p(x_i|\textbf{x}_{-i})}{1-p(x_i|\textbf{x}_{-i})} = \theta_1 \sum_{i=1}^m x_i + \beta_1 \sumnbr x_ix_{i'}
\end{equation} 

\begin{equation}
\frac{p(x_i|\bm{x}_{-i})}{1-p(x_i|\bm{x}_{-i})} = \theta_1 \sum_{i=1}^m x_i + \beta_1 \sumnbr x_ix_{i'}
\end{equation} 

Test the above equation label in here (\ref{eq:logodds}).

\begin{eqnarray}
(x+y)^2 & = & x^2 + xy + yx + y^2 \label{eqnarray1} \\
		& = & x^2 + xy + xy + y^2 \nonumber \\
		& = & x^2 + 2xy + y^2 \label{eqnarray2}
\end{eqnarray}

\begin{eqnarray}
(x+y)^2 & = & x^2 + xy + yx + y^2 \\
		& = & x^2 + 2xy + y^2 \nonumber
\end{eqnarray}

\begin{eqnarray*}
(x+y)^2 & = & x^2 + xy + yx + y^2 \\
		& = & x^2 + 2xy + y^2 \nonumber
\end{eqnarray*}

\begin{eqnarray*}
\lefteqn{a+b+c+d+e+f+g+h+i+j+k+l=} \\
& & x+y+z+a+b+c+d+e+f+g+o+s+t+ \\
& & u+v+w
\end{eqnarray*}

Math italic $different$ is from \emph{different}.

\begin{equation}
f(x) = \left \{ \begin{array}{rl}
x & if x > 2 or if x < -2 \\
x & {\rm if} x > 2 {\rm or if} x < -2 \\
x & \mbox{if} x > 2 \mbox{ or if } x < -2
\end{array} \right.
\end{equation}

$\mathrm{Form\ e^{pdf}+ \phi(x)}$ \\

$\mathit{Form\ e^{pdf}+\phi(x)}$ \\

$\mathtt{Form\ e^{pdf}+\phi(x)}$ \\

$\mathbf{Form\ e^{pdf}+\phi(x)}$ \\

$\mathsf{Form\ e^{pdf}+\phi(x)}$ \\

$\mathcal{ABC}$ \\

${\bf a} = (a_1, a_2, \ldots, a_n)^T$ \\

\boldmath{$a = (a_1, a_2, \ldots, a_n)^T$}\unboldmath \\

\boldmath{$a \mbox{\unboldmath$=(a_1, a_2, \ldots, a_n)^T$}$} \unboldmath \\

$\bm{a}=(a_1, a_2, \ldots, a_n)^T$ \\

$\pmb{aX + \beta + \gamma}$ \\

$\boldsymbol{aX + \beta + \gamma}$ \\

$A\stackrel{f}{\to}B\stackrel{g}{\to}C$ \\ 

$A \small{A} \LARGE{A} \Huge{A} \footnotesize{A}$ \\

${\displaystyle \frac{a+b}{c-d}}$ \\

$\frac{\textstyle a+b}{\displaystyle a+b}$ \\

$\displaystyle 1+\frac{1}{1+\frac{1}{1+\frac{1}{1+\frac{1}{1+\frac{1}{1+x}}}}}$ \\

$1+\frac{1}{1+\displaystyle \frac{1}{1+\displaystyle \frac{1}{1+\displaystyle \frac{1}{1+\displaystyle +\frac{1}{1+x}}}}}$ \\

\begin{equation}
\mbox{표준편차} = \sqrt{\mbox{분산}} = \sqrt{\frac{\mbox{편차}^2 \mbox{의 합}}{\mbox{표본의 개수} - 1}} \\
\end{equation}

$\int_0^\infty f(x) dx$ \\

$\int \limits_0^\infty f(x) dx$ \\

${\displaystyle \int \limits_0^\infty f(x) dx}$ \\ 

$\int \nolimits_0^\infty f(x) dx$ \\

${\displaystyle \int_0^\infty f(x) dx}$ \\

$\infsup_{n\rightarrow \infty} f_n(x)$ \\

$\displaystyle \infsup_{n\rightarrow \infty} f_n(x)$ \\

$\infsup \limits_{n\rightarrow \infty}^{\rm woops} f_n(x)$ \\

$\infsup \nolimits_{n\rightarrow \infty}^{\rm woops} f_n(x)$ \\

$2n \choose n$, ${}_{2n}\rm C_n$ \\

$\left(\begin{array}{c} 2n \\ n \\ \end{array}\right)$ \\

$(_{\ n\,}^{2n})$ \\

$x \brack 2y$ \\

$a-c \brace b$ \\

${}_{ 73}^{231}{\rm U}$ \\

${}_{\phantom{0}73}^{231}{\rm U}$ \\

$x+{a+b \atopwithdelims<> c}$ \\

$x+{a+b \atopwithdelims\uparrow\downarrow c}$ \\

$A=\begin{pmatrix}
\lambda & l \cr 
a & \alpha \cr
\end{pmatrix}$ \\

$B=\left(\begin{array}{cc}
\lambda & 1 \\
a & \alpha
\end{array} \right)$

$$A=\bordermatrix{& n_1 & n_2 & n_3 \cr
m_1 & A_{11} & A_{12} & A_{13} \cr
m_2 & A_{21} & A_{22} & A_{23} \cr}$$ \\

\[ 
A = \begin{array}{cccc} & n_1 & n_2 & n_3 \\
\begin{array}{c} m_1 \\ m_2 \end{array} &
\left(
\begin{array}{c} A_{11} \\ A_{21} \end{array} \right. &
\begin{array}{c} A_{12} \\ A_{22} \end{array} & \left.
\begin{array}{c} A_{13} \\ A_{23} \end{array} \right)
\end{array} 
\] \\

%$f(x)= \begin{cases}x & for $x>0$ \cr -x & for $-1<x \leq 0$ \cr x^2 & otherwise \cr \end{cases}$

$$f(x)=\left\{\begin{array}{ll} x & \mbox{for } x > 0 \\ -x & \mbox{for } -1 < x \leq 0 \\ x^2 & \mbox{otherwise} \end{array} \right.$$ \\

$$f(x)=\left\{\begin{array}{rl} x & \mbox{for } x > 0 \\ -x & \mbox{for } -1<x\leq 0 \\ x^2 & \mbox{otherwise}
\end{array} \right.$$ \\

${a+b \over c+d}$ $\frac{a+b}{c+d}$ \\

$\{ x\Big | x\in X\}$ \\

$\{x\Bigm | x\in X\}$ \\

$\{x \Bigl | x\in X\}$ \\

$\{x \Bigr | x\in X\}$ \\

$\Biggl \{ \biggl \{ \Bigl \{ \bigl \{ \{ \} \bigr \} \Bigr \} \biggr \} \Biggr \}$ \\

$|b=|x+y||$ \\

$\Bigl | b-|x+y| \Bigr |$

%\begin{subequations}
%\begin{align}
%& \frac{1}{\sqrt{2\pi}\sigma}\exp\{-\frac{(x-\mu)^2}{2\simga^2}\}\mbox{ if } \mu=0, \sigma=1, \label{eq:subeq1} \\
%& \frac{1}{\sqrt{2\pi}\sigma}\exp\{-\frac{x^x}{2\sigma^2}\}\mbox{ if } \mu=0, \label{eq:subeq2} \\
%& \frac{1}{\sqrt{2\pi}} \exp\{-\frac{x^2}{2}\} \mbox{ if } \mu=0, \sigma=1. \label{eq:subeq3}
%\end{align}
%\end{subequations}

\begin{subequations}
\begin{align}
& \frac{1}{\sqrt{2\pi}\sigma}\exp\{-\frac{(x-\mu)^2}{2\sigma^2}\}\mbox{ if } \mu=0, \sigma=1, \label{eq:subeq1} \\
& \frac{1}{\sqrt{2\pi}\sigma}\exp\{-\frac{x^x}{2\sigma^2}\}\mbox{ if } \mu=0, \label{eq:subeq2} \\
& \frac{1}{\sqrt{2\pi}} \exp\{-\frac{x^2}{2}\} \mbox{ if } \mu=0, \sigma=1. \label{eq:subeq3}
\end{align}
\end{subequations}

식 (\ref{eq:subeq2})와 식 (\ref{eq:subeq3})\는 식 (\ref{eq:subeq1})의 특별한 경우이다.

$\mathbb{A B C D E R}$ \\

$\mathfrak{Mathfrack Font}$ \\

$\circledR \checkmark \maltese$ \\

$\digamma \varkappa \beth \daleth \gimel$ \\

$\ulcorner \urcorner \llcorner \lrcorner$ \\

$$\sum_{\substack{i=j\\ i<k \\ j\neq k}}x_{ij}^{j-k}$$

$$\iiint \limits_{\substack{x\in A \\ y \in B \\ c \ni z}} f(x, y, z) dxdydz$$

$$\sum^{\sum^{\substack{n \\ m-n }}} x_{ij}$$

\begin{multline} \label{eq:multline}
a = d+e+f+g+h+i+j+k+l+m\\
+q+r+s+t+u+v+w+x+y+x+z\\
\shoveleft{+q+r+s+t+u+v+w+x+y+x+z}\\
=e+f+g+h+i+j+k+l+m+z+y+x+m_2\\
\end{multline}

\begin{equation}
\begin{split}
a &= b + c + d + e \\
  &- f + g + m \\
  &= x^2 + y^2+z
\end{split}
\end{equation}

\begin{gather}
(x+y)^2=x^2+2xy+y^2\\
(x+y+z)^2=x^2+y^2+z^2+2xy+2yz+2zx
\end{gather}

\begin{align}
(x+y)^2 &= x^2+2xy+y^2 \\
(x+y+z)^2 &= x^2 + y^2 + z^2 + 2xy + 2yz + 2zx
\end{align}

\begin{alignat}{2}
a_{11} &= a_{12} & a_{13} &= a_{14} \\
(x+y)^2 &= (x+y)(x+y) & x^2+2xy+y^2 &= z^2 \\
f(x) &= \dfrac{1}{(1+x^2)} & g(x) &= \sqrt{2\pi}
\end{alignat}

\begin{equation} 
\begin{aligned}
a &= \alpha \\
b &= \beta \beta \\
c &= \gamma\gamma\gamma
\end{aligned} \qquad \text{versus} \qquad
\begin{aligned}
D &= \Delta\Delta\Delta \\
e &= \epsilon\epsilon \\
Z &= \Omega
\end{aligned}
\end{equation}

\begin{equation} \label{eq:gathered}
\begin{gathered}
a = \alpha \\
b = \beta\beta \\
c = \gamma\gamma\gamma \\
\end{gathered} 
\qquad \text{versus} \qquad
\begin{gathered}
D = \Delta\Delta\Delta \\
e = \epsilon\epsilon \\
Z = \Omega
\end{gathered} 
\end{equation}

\end{document}