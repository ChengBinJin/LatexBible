\documentclass[11pt]{article}

\usepackage{kotex}

\newcommand{\dsum}[2]{\displaystyle \sum_{#1}^{#2}}
\newcommand{\vphi}{\varphi}
\newcommand{\go}{\longrightarrow}
\newcommand{\Go}{\Longrightarrow}
\newcommand{\Hajek}{Hajek}
\newcommand{\Sidak}{Sidak}
\newcommand{\dlim}[1]{{\displaystyle \lim_{#1}}}
\newcommand{\harcombi}[2]{{}_{#1}{\rm H}_{#2}}
\newcommand{\permut}[2]{{}_{#1}{\rm P}_{#2}}
\newcommand{\combi}[2]{{}_{#1}{\rm C}_{#2}}

\newcommand{\tmp}[1][0]{$\mathrm{R}^{#1}$}
\renewcommand{\tmp}[2][0]{$#2^#1$}
\providecommand{\mdegree}[1]{#1\ensuremath{^\circ}}
\newcommand{\AL}{\ensuremath{\alpha}}
\newcommand{\htex}{\mbox{한글\TeX}}
\newcommand{\HandS}{\Hajek and \Sidak}
\newcommand{\pandc}[2]{\permut{#1}{#2} \ne \combi{#1}{#2}}
\newtheorem{theorem}{Theorem}[section]
\newtheorem{la}{\scshape Lemma}[section]

\begin{document}
$\dsum{i=1}{n} x_i$ is the same as $\sum_{i=1}^n x_i$.

Hajek and Sidak proved that $\vphi (i/n)\go \vphi(u)$ as $n\go\infty$. They used  ''user-defined'' displaystyle $\dlim{n\go\infty}$ rather than textstyle $\lim_{n\go\infty}$.

순열, 조합, 중복조합 간에는 다음 관계가 있다.
\begin{enumerate}
\item $\permut{n}{r}=n(n-1)\cdots(n-r+1)$
\item $\combi{n}{r}={\displaystyle \frac{\permut{n}{r}}{r!}}$
\item $\harcombi{n}{r}=\combi{n+r-1}{r}$
\end{enumerate}

\tmp[5] \\

\tmp{0} \\

\tmp[3]{x} \\

\tmp{x}  \\

\mdegree{360}C나 $\mdegree{360}$C는 같은 결과

새로 선안한 \verb+\AL+은 \AL나 $\AL$와 같이 두 모드에서 모두 사용할 수 있다.

\htex 은 한글을 지원한다.

\HandS

$\pandc{n}{r}$

\begin{theorem}[\LaTeX] \label{th:texusage}
\LaTeX\ is easy enough to use for all scientists.
\end{theorem}

\begin{la}
This is a test Lemma.
\end{la}

\end{document}