\documentclass[11pt]{article}

\usepackage{kotex}

\newcommand{\dsum}[2]{\displaystyle \sum_{#1}^{#2}}
\newcommand{\vphi}{\varphi}
\newcommand{\go}{\longrightarrow}
\newcommand{\Go}{\Longrightarrow}
\newcommand{\Hajek}{H\'{a}jek}
\newcommand{\Sidak}{S\v{\id}d\'s{a}k}
\newcommand{\dlim}[1]{{\displaystyle \lim_{#1}}}
\newcommand{\harcombi}[2]{{}_{#1}{\rm H}_{#2}}
\newcommand{\permut}[2]{{}_{#1}{\rm P}_{#2}}
\newcommand{\combi}[2]{{}_{#1}{\rm C}_{#2}}

\begin{document}
$\dsum{i=1}{n} x_i$ is the same as $\sum_{i=1}^n x_i$.

Hajek and Sidak proved that $\vphi (i/n)\go \vphi(u)$ as $n\go\infty$. They used  ''user-defined'' displaystyle $\dlim{n\go\infty}$ rather than textstyle $\lim_{n\go\infty}$.

순열, 조합, 중복조합 간에는 다음 관계가 있다.
\begin{enumerate}
\item $\permut{n}{r}=n(n-1)\cdots(n-r+1)$
\item $\combi{n}{r}={\displaystyle \frac{\permut{n}{r}}{r!}}$
\item $\harcombi{n}{r}=\combi{n+r-1}{r}$
\end{enumerate}

\end{document}