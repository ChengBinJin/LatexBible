\documentclass[11pt]{article}

\usepackage{kotex}
\usepackage{amsmath}

\newcommand{\dsum}[2]{{\displaystyle \sum_{#1}^{#2}}}
\newcommand{\vphi}{\varphi}
\newcommand{\go}{\longrightarrow}
\newcommand{\Go}{\Longrightarrow}
\newcommand{\Hajek}{H\'{a}jek}
\newcommand{\Sidak}{S\v{\i}d'{a}k}
\newcommand{\dlim}[1]{{\displaystyle \lim_{#1}}}

\newcommand{\harcombi}[2]{{}_{#1}{\rm H}_{#2}}
\newcommand{\permut}[2]{{}_{#1}{\rm P}_{#2}}
\newcommand{\combin}[2]{{}_{#1}{\rm C}_{#2}}

\newcommand{\tmp}[1][0]{$\mathrm{R}^{#1}$}
\renewcommand{\tmp}[2][0]{$#2^#1$}
\newcommand{\AL}{\ensuremath{\alpha}}
\newcommand{\htex}{\mbox{한글\TeX}}

\providecommand{\mdegree}[1]{#1\ensuremath{^\circ}}
\newcommand{\HandS}{\Hajek \ and \Sidak}
\newcommand{\pandc}[2]{\permut{#1}{#2} \ne \combin{#1}{#2}}

\newtheorem{theorem}{Theorem}
\newtheorem{lemma}{Lemma}
\newtheorem{coro}{Corollary}
\newtheorem{defn}{Definition}
\newtheorem{syntax}{문법}
\newenvironment{mytheorem}{\begin{theorem} \normalfont}{\end{theorem}}

\newenvironment{myfbox}[1][\textwidth]%
{\begin{tabular}{|@{\hspace{1em}}c@{\hspace{1em}}|}%
\hline \\*[-.3\baselineskip]
\begin{minipage}[t]{#1}}%
{\end{minipage} \\ \vspace{-.3\baselineskip} \\
\hline \end{tabular}}

\newenvironment{barray}[1]{\left[\begin{array}{#1}}{\end{array}\right]}

\begin{document}
$\dsum{i=1}{n} x_i$ is the same as $\sum_{i=1}^n x_i$.

\Hajek and \Sidak proved that $\vphi(i/n)\go\vphi(u)$ as $n\go\infty$. They used ``user--defined'' displaystyle $\dlim{n\go\infty}$ rather than textstyle $\lim_{n\go\infty}$.

순열, 조합, 중복조합 간에는 다음 관계가 있다.
\begin{enumerate}
\item $\permut{n}{r}=n(n-1)\cdots(n-r+1)$
\item $\combin{n}{r}={\displaystyle \frac{\permut{n}{r}}{r!}}$
\item $\harcombi{n}{r}=\combin{n+r-1}{r}$
\end{enumerate}

\tmp[3]{x}

\tmp{x}

\mdegree{360}C나 $\mdegree{360}$C는 같은 결과

새로 선언한 \verb+\AL+은 \AL 나 $\AL$와 같이 두 모드에서 모두 사용할 수 있다.

\htex 은 한글을 지원한다.

\HandS

$\pandc{n}{r}$

\begin{theorem}[\LaTeX] \label{th:texusage}{\  \\
\LaTeX\ is easy enough to use for all scientists}
\end{theorem}

\begin{large}
This is a test Lemma.
\end{large}

\begin{mytheorem}[\LaTeX]
{\ \\
\LaTeX\ is easy enough to use for all scientists}
\end{mytheorem}

\begin{myfbox}[4.5in]
\vspace*{-1.5\baselineskip}
\begin{theorem}[\LaTeX] \label{th:texsage2}
\LaTeX\ is easy enough to use for all scientists. And it can define new environment within itself.
\end{theorem}
\vspace*{-1.5\baselineskip}
\end{myfbox}

\begin{myfbox}[.36\textwidth]
This chapter covers that defining new command, theorem and the like, and environment
\end{myfbox} WHICH
\begin{myfbox}[.36\textwidth]
makes your customized \LaTeX\ job easy and fun. Enjoy your \LaTeX.
\end{myfbox}

$$\mbox{{\boldmath $y$}}=\mbox{{\boldmath $Ab$}} = \begin{barray}{cc} a_{11} & a_{12} \\ a_{21} & a_{22} \end{barray}
\begin{barray}{c} b_1 \\ b_2 \\ \end{barray}$$

\newenvironment{example}
{\vspace*{.6\baselineskip}\hrule\nopagebreak[4]
\begin{list}{}
{\setlength{\rightmargin}{.2in}}
\refstepcounter{excntr}
\item[예 \thechapter.\arabic{excntr}]}
{\vspace*{.6\baselineskip}\hrule\end{list}}

\begin{verbatim}
\begin{example}
\newtheorem{lemma}{Lemma}|
\newtheorem{theorem}{Theorem}[chapter]|
\newtheorem{coro}{Corollary}|
\newtheorem{defn}{Definition}|
\newtheorem{syntax}{문법}[chapter]
\end{example}
\end{verbatim}



\end{document}