\documentclass[10pt]{article}

\usepackage{kotex}

\addtolength{\textwidth}{2cm}
\addtolength{\textheight}{2cm}
\addtolength{\evensidemargin}{-1cm}
\addtolength{\oddsidemargin}{-1cm}
%\addtolength{\topsidemargin}{-1cm}

\renewcommand{\baselinestretch}{1.6}
\addtolength{\baselineskip}{2pt}

\begin{document}
Hello world!

여기에 공\hspace{1in}간.

여기에 공\hspace{-3em}간이 있다.

여기에 공\hspace{0.1\textwidth}간이 있다. 

\hspace{.5\textwidth} 이 앞에 공간이 있다.


\hspace*{.5\textwidth} 이 앞에 공간이 있다.

여기에 세로 공\vspace{1cm}간을 만들려고 하면 문장의 중간이기
때문에 예상 외의 결과를 얻는다. 즉, 한 줄을 완성한 후에 공간을
만든다. 여기에 세로 공\\ \vspace{1cm}간을 만들려고 하면
문장의 중간이라도 행이 바뀐 뒤이기 때문에 가능하다.

``여기에 세로 공\\*[1cm]간을 만들력고 하면 ..."

여기에서 바로 줄\vskip1.em 바꾸기가 됩니다.

이 문장은 \hspace{\fill} 양쪽 끝에 있다. \\
이 문장은 \hspace{\fill} 세 부분 \hspace{\fill}으로 이루어진다.

\verb|\dotfill|의\dotfill 효과

\verb+\dotfill+의 \dotfill 효과

\verb+\hrulefill+의 \hrulefill 효과 \hrulefill 데모

\hfill \today \\

\null \hfill \today \\

\makebox[2cm]{2cm}의 박스

\makebox[2cm][l]{2cm}의 박스

\makebox[2cm][r]{2cm}의 박스

여기에 \mbox{3cm} 의 박스

여기에 \framebox[1in]{사각형}이 있다.

여기에 \framebox[1in][l]{사각형}이 있다.

O O O \framebox[1cm]{This is a box} O O O

O O O \makebox[1cm]{This is a box} O O O

\fbox{김철수}

\fbox{\fbox{김철수}}

\fbox{go}, \fbox{enter}, \fbox{joy}

\fbox{\strut go}, \fbox{\strut enter}, \fbox{\strut joy}

\parbox{2in}{A box of 2 cm length is here
and a second box will be next} \ \hfill BETWEEN \ \hfill
\parbox{3cm}{This box has 3cm length} \\ 

여기에 \hfill \fbox{\parbox{3in}{여기에 박스를 넣으면
박스 앞뒤로 다른 문장이 이어져 나온다. 그 이유는 줄을
바꾸지 않았기 때문이다.}} \hfill 박스가 있다. \\

\begin{minipage}[t]{2in}
\texttt{minipage}는 \texttt{parbox}와 비슷하다.
반드시 폭을 주어야 한다.
\end{minipage} 
\textbf{그러나} \ 
\begin{minipage}[b]{2in}
위치는 줄 수도 있고 주지 않을 수도 있다. 지정해 주지 않으면 \texttt{c}\footnote{중앙(center)을 의미}를 가정한다.
\end{minipage}

\marginpar{\footnotesize 오른쪽에 만들어진 가장자리 문단}

\addtolength{\marginparwidth}{-1cm}
\reversemarginpar
\marginpar{왼쪽에 만들어진 가장자리 문단}

\normalmarginpar
\addtolength{\marginparwidth}{1cm}

\end{document}