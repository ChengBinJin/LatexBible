\documentclass[11pt]{article}

\usepackage{kotex}

\addtolength{\evensidemargin}{-1cm}
\addtolength{\oddsidemargin}{-1cm}
%\addtolength{\topsidemargin}{-1cm}

\begin{document}
여기에 공\hspace{1in}간.

여기에 공\hspace{-3em}간이 있다.

여기에 공\hspace{.1\textwidth}간이 있다. 

"\verb|\hspace*{.5\textwidth}| 여기에 공간이 있다."

\hspace*{.5\textwidth} 여기에 공간이 있다.

여기에 세로 공\vspace{1cm}간을 만들려고 하면 문장의 중간이기 때문에 예상 외의 결과를 얻는다. 즉, 한 줄을 완성한 후에 공간을 만든다. 여기에 세로 공\\ \vspace{1cm}간을 만들려고 하면 문장의 중간이라도 행이 바뀐 뒤이기 때문에 가능하다.

``여기에 세로 공\\*[1cm]간을 만들려고 하면 ...'' 의 결과는

여기에서 바로 줄\vskip1.em 바꾸기가 됩니다.

이 문장은 \hspace{\fill} 양쪽 끝에 있다. 

이 문장은 \hspace{\fill} 세 부분 \hspace{\fill}으로 이루어진다. \\

\vfill의 예로 만든 공간으로 설명 목적상 페이지의 중간에 공간이 있습니다.

\verb|\dotfill|의 \dotfill효과

\verb|\hrulefill|의 \hrulefill효과 \hrulefill데모


\end{document}