\documentclass[10pt]{article}

\usepackage{kotex}

\addtolength{\textwidth}{2cm}
\addtolength{\textheight}{2cm}
\addtolength{\evensidemargin}{-1cm}
\addtolength{\oddsidemargin}{-1cm}
%\addtolength{\topsidemargin}{-1cm}

\renewcommand{\baselinestretch}{1.6}
\addtolength{\baselineskip}{2pt}

\begin{document}
Hello world!

여기에 공\hspace{1in}간.

여기에 공\hspace{-3em}간이 있다.

여기에 공\hspace{0.1\textwidth}간이 있다. 

\hspace{.5\textwidth} 이 앞에 공간이 있다.


\hspace*{.5\textwidth} 이 앞에 공간이 있다.

여기에 세로 공\vspace{1cm}간을 만들려고 하면 문장의 중간이기
때문에 예상 외의 결과를 얻는다. 즉, 한 줄을 완성한 후에 공간을
만든다. 여기에 세로 공\\ \vspace{1cm}간을 만들려고 하면
문장의 중간이라도 행이 바뀐 뒤이기 때문에 가능하다.

``여기에 세로 공\\*[1cm]간을 만들력고 하면 ..."

여기에서 바로 줄\vskip1.em 바꾸기가 됩니다.

이 문장은 \hspace{\fill} 양쪽 끝에 있다. \\
이 문장은 \hspace{\fill} 세 부분 \hspace{\fill}으로 이루어진다.

\verb|\dotfill|의\dotfill 효과

\verb+\dotfill+의 \dotfill 효과

\verb+\hrulefill+의 \hrulefill 효과 \hrulefill 데모

\hfill \today \\

\null \hfill \today \\



\end{document}