\documentclass[11pt]{article}

\usepackage{kotex}

\renewcommand{\labelitemi}{$\clubsuit$}
%\renewcommand{\labelitemii}{$\fingerright$}
\renewcommand*\descriptionlabel[1]{\hspace\labelsep\scshape\mdseries #1}
\renewcommand{\labelenumi}{\onum{enumi}}
\renewcommand{\labelenumii}{\ogana{enumii}}

\begin{document}
In this section we study following two....
\begin{itemize}
\item First item
\item Second item
\end{itemize}

To explain \texttt{itemize}, we have two items below;
\begin{itemize}
\item First item got subitems
\begin{itemize}
\item This is first subitem
\item Second subitem
\end{itemize}
\item Second item
\end{itemize}

\begin{itemize}
\item[First] First item
\item[Second] Second item
\item[Third] Third item
\end{itemize}

\begin{description}
\item[First] First item
\item[Second] Second item
\end{description}

\begin{description}
\item First item
\item Second item
\end{description}

\begin{description}
\item[First] First item
\item[Second] Second item
\end{description}

\begin{enumerate}
\item First item
\begin{enumerate}
\item First subitem
\end{enumerate}
\item Second item
\end{enumerate}

\begin{enumerate}
\item 첫째 항
\begin{enumerate}
\item 첫째의 첫째 항
\item 첫째의 둘째 항
\end{enumerate}
\item 둘째 항
\item 세 번째 항
\end{enumerate}





\end{document}