\documentclass{beamer}
%\usepackage{syshan}
\usepackage{kotex}
\usepackage{colortbl}
\usepackage{tikz}
%\usepackage{graphicx}

\usetheme{default}
%\usetheme{Madrid}
%\usetheme{Antibes}
%\usetheme{JuanlesPins}
%\usetheme{Bergen}
%\usetheme{Boadilla}  % like this theme
%\usetheme{Pittsburgh}
%\usetheme{Montpellier}
%\usetheme{PaloAlto}
%\usetheme{Goettingen}
%\usetheme{Marburg}
%\usetheme{Hannover}
%\usetheme{Berlin}
%\usetheme{Ilmenau}
%\usetheme{Dresden}
%\usetheme{Darmstadt}
%\usetheme{Frankfurt}
%\usetheme{Singapore}
%\usetheme{Szeged}
%\usetheme{Copenhagen}
%\usetheme{Luebeck}
%\usetheme{Malmoe}
%\usetheme{Warsaw}

%\usecolortheme{albatross}
%\usecolortheme{beaver}
%\usecolortheme{beetle}
%\usecolortheme{crane}
%\usecolortheme{default}
%\usecolortheme{dolphin}
%\usecolortheme{dove}
%\usecolortheme{fly}
%\usecolortheme{lily}
%\usecolortheme{orchid}
%\usecolortheme{rose}
%\usecolortheme{seagull}
%\usecolortheme{seahorse}
%\usecolortheme{sidebartab}
%\usecolortheme{structure}
\usecolortheme{whale}  % like this
%\usecolortheme{wolverine}

%\usefonttheme{professionalfonts}
%\usefonttheme{default}
%\usefonttheme{serif}
\usefonttheme{structurebold}  % like this
%\usefonttheme{structuresmallcapsserif}

\useinnertheme{circles}  % like this
%\useinnertheme{default}
%\useinnertheme{inmargin}
%\useinnertheme{rectangles}
%\useinnertheme{rounded}

%\useoutertheme{default}
%\useoutertheme{infolines}
%\useoutertheme{miniframes}
\useoutertheme{shadow}  % like this
%\useoutertheme{sidebar}
%\useoutertheme{smoothbars}
%\useoutertheme{split}
%\useoutertheme{tree}

\title{중심극한정리}
\subtitle{$n$이 클 때의 $\bar{X}$의 분포}
\author[Cheng-Bin Jin]{\texttt{chengbinjin@inha.edu}}
\date[\today]{한국통계학회 2007년 추계학술대회}
\institute{인하대 정보통신학과}

\begin{document}
\setbeamercovered{transparent=25}

\begin{frame}
\titlepage
\end{frame}

\begin{frame}
\frametitle{Central Limit Theorem}
\begin{theorem}<1->
$X_1, X_2, \cdots, X_n$이 독립이고 $E[X] = \mu$,
$Var(X)=\sigma^2 (>0)$ 이고 $\bar{X}$와 $S^2$이 각각
표본평균, 표본분산이라고 하자.
이때 $n \rightarrow \infty$ 이면
\begin{equation} \label{1}
\frac{\bar{X} - \mu}{S/\sqrt{n}} \stackrel{asymp.}{\sim}N(0,1)
\end{equation}
이다.
\end{theorem}
\begin{proof}[Sketch of proof]<2->
\begin{enumerate}
\item<3-> 식 (\ref{1})의 $ch.f$ $\phi(t)$의 expansion을 구한다.
\item<4-> $n \rightarrow \infty$일 때 이 함수가 수렴함을 보인다.
\end{enumerate}
\end{proof}
\end{frame}

\begin{frame}
\begin{columns}[T]
\begin{column}{3in}
\centerline{
\includegraphics[width=2in, height=3in]{../images/mari.png}}
\end{column}
\begin{column}{8cm}<2->
\begin{itemize}
\item 통계의 마술
\item 5\% 의 진정한 의미
\item 등등
\end{itemize}
\end{column}
\end{columns}
\end{frame}

\begin{frame}
\frametitle{Row-wise display}
\centerline{
\begin{tabular}{l!{\vrule}rrrrr}
\rowcolor[rgb]{.9,0,0} age group & 10s & 20s & 30s & 40s & 합 \\ \hline
\rowcolor[gray]{.7} \pause 남 & 10 & 5 & 78 & 45 & 138 \\
\rowcolor[rgb]{0, .9, 0} \pause 여 & 15 & 34 & 2 & 12 & 63 \\
\rowcolor[rgb]{0, 0, .9} \pause 합 & 25 & 39 & 80 & 57 & 201\\ \hline
\end{tabular}}
\end{frame}

\begin{frame}
\frametitle{Column-wise display}
\begin{tabular}{l| r<{\onslide<2->} r<{\onslide<3->}
r<{\onslide<4->} r<{\onslide<4->} r<{\onslide<5>}r} \hline
age group & 10s & 20s & 30s & 40s & 합 \\ \hline
남 & 10 & 5 & 78 & 45 & 138 \\
여 & 15 & 34 & 2 & 12 & 63 \\ \hline
합 & 25 & 39 & 80 & 57 & 201 \\ \hline
\end{tabular}
\end{frame}

\begin{frame}
\frametitle{\texttt{tikz} 패키지}
\begin{center}
\begin{tikzpicture}[yscale=5, xscale=1]
\node<1->[red] at (1, 0.6) {정규분포 곡선};
node[below=0.8cm, white] at (1,0)
{$f(x)=\frac{1}{\sqrt{2\pi}}
\Big\{-\frac{(x-\mu)^2}{2\sigma^2}\Big\}$};
\draw<1->[thick,->](-3.5,0) -- (5.5,0);
\draw<1->[black, domain=-3:3]
plot(\x, {1/sqrt(2*pi)*exp(-pow(\x,2)/2)})
node[below] at (0, 0) {\small $\mu=0$};
\draw<1->[dotted] (0,0) -- (0,0.3989423); %평균선
%% 두 번째 슬라이드
\onslide<2->{
\draw[red, domain=-2:4]
plot(\x, {1/sqrt(2*pi)*exp(-pow(\x-1,2)/2)})
node[below] at (1,0) {\small $\mu=1$};
\draw[red, dotted] (1,0) -- (1, 0.3989423); %평균선
}
%% 세 번째 슬라이드
\only<3->{
\draw[blue, domain=-1:5]
plot(\x, {1/sqrt(2*pi)*exp(-pow(\x-2,2)/2)})
node[below] at (2,0) {\small $\mu=2$};
\draw[blue, dotted] (2,0) -- (2,0.3989423); %평균선
}
%% 마지막 슬라이드
\draw<4> node[below=0.8cm] at (1,0)
{$f(x)=\frac{1}{\sqrt{2\pi}}
\Big\{-\frac{(x-\mu)^2}{2\sigma^2} \Big\}$};
\end{tikzpicture}
\end{center}
\end{frame}





\end{document}