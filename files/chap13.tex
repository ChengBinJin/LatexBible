\documentclass{beamer}

\usepackage{multimedia}
\usepackage{kotex}

\usetheme{Madrid}

\title{Beamer 보기}
\subtitle{Overlay 방법들}
\author[Cheng-Bin Jin]{\texttt{sbkim0407@gmail.com}}
\date[\today]{인하대학교 정보통신학과}
\institute{컴퓨터비젼 연구실}

\begin{document}

\begin{frame}
\titlepage
\end{frame}

\setbeamercovered{transparent=20}
\begin{frame}
\frametitle{pause 보기}
The first slide \\
\pause
두 번쨰 화면 \\
\pause
The third slide \\
\pause[2]
두 번쨰 이후에는 계속 보입니다. \\
\pause
The last slide \\
\end{frame}

\begin{frame}
\frametitle{onslide 보기-공간 차지}
\onslide<1>{첫 번째 슬라이드에만 보입니다.\\}
\onslide<2>{두 번째 슬라이드에만 보입니다.\\}
\onslide<3->{세 번째 슬라이드 이후 보입니다.\\}
\onslide<4->{네 번째 슬라이드에만 보입니다.}
\end{frame}


\end{document}