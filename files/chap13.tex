\documentclass{beamer}

\usepackage{multimedia}
\usepackage{kotex}

\usetheme{Madrid}

\title{Beamer 보기}
\subtitle{Overlay 방법들}
\author[Cheng-Bin Jin]{\texttt{sbkim0407@gmail.com}}
\date[\today]{인하대학교 정보통신학과}
\institute{컴퓨터비젼 연구실}

\begin{document}

\begin{frame}
\titlepage
\end{frame}

\setbeamercovered{transparent=20}
\begin{frame}
\frametitle{pause 보기}
The first slide \\
\pause
두 번쨰 화면 \\
\pause
The third slide \\
\pause[2]
두 번쨰 이후에는 계속 보입니다. \\
\pause
The last slide \\
\end{frame}

\begin{frame}
\frametitle{onslide 보기-공간 차지}
\onslide<1>{첫 번째 슬라이드에만 보입니다.\\}
\onslide<2>{두 번째 슬라이드에만 보입니다.\\}
\onslide<3->{세 번째 슬라이드 이후 보입니다.\\}
\onslide<4->{네 번째 슬라이드에만 보입니다.}
\end{frame}

\newcommand{\blueonly}{\only{\color{blue}}}
\begin{frame}
\blueonly<2> 두 번째 슬라이는?
\end{frame}

\begin{frame}
\frametitle{only 보기-공간 없음}
\only<1>{첫 번째 슬라이드에만 보입니다.\\}
\only<2>{두 번째 슬라이드에만 보입니다.\\}
\only<3>{세 번쨰 슬라이드 이후 보입니다.\\}
\only<4>{네 번째 슬라이드에만 보입니다.}
\end{frame}

\begin{frame}
\frametitle{\texttt{alt} 보기-공간 없음}
\begin{itemize}
\item<1>{첫 번째 슬라이드에만 보입니다. \newline}
\item<2> \alt<2>{두 번째 슬라이드에만 보입니다. \newline}
{네 번째 슬라이드에만 보입니다. \newline}
\item<3->{세 번째 슬라이드 이후 보입니다. \newline}
\item<4>{네 번째 슬라이드에만 보입니다.}
\end{itemize}
\end{frame}

\setbeamercovered{highly dynamic}
\begin{frame}
\frametitle{\texttt{temporal} 보기-공간 있음}
\onslide<1>{첫 번째 슬라이드에만 보입니다.\\}
\temporal<2>{Before 2nd slide\\}
{두 번째 슬라이드에만 보입니다. \\}
{After 2nd slide \\}
\onslide<3->{세 번째 슬라이드 이후 보입니다.\\}
\onslide<4>{네 번째 슬라이드에만 보입니다.}
\end{frame}

\begin{frame}
\frametitle{자동으로 보이기}
\begin{itemize}[<+-| alert@+>]
\item $\int_0^1 f(x) dx$
\item $S = \sum_{i=1}^n a_i$
\item 세 번째 항목
\item The last item
\end{itemize}
\end{frame}

\begin{frame}
\frametitle{enumerate 보기}
\begin{enumerate}[<+->][I:]
\item 2, 4, 5 는 삼각형이다.
\item 2, 3, 5 는 삼각형이 될 수 없다.
\item 7, 4, 5 는 삼각형이다.
\end{enumerate}
\end{frame}

\begin{frame}
\frametitle{description 보기}
\begin{description}[<+->][제일 긴 항목]
\item[항목 1] 2, 4, 5 는 삼각형이다.
\item[제일 긴 항목] 2, 3, 5 는 삼각형이 될 수 없다.
\item[항목 3] 7, 4, 5 는 삼각형이다.
\end{description}
\end{frame}

\begin{frame}
\frametitle{description 보기}
\begin{description}[<+->][항목 1]
\item[항목  1] 2, 4, 5 는 삼각형이다.
\item[제일 긴 항목] 2, 3, 5 는 삼각형이 될 수 없다.
\item[항목 3] 7, 4, 5 는 삼각형이다.
\end{description}
\end{frame}




\end{document}