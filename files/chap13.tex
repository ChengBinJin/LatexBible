\documentclass[xcolor=table]{beamer}

\usepackage{multimedia}
\usepackage{kotex}
%\usetheme{Madrid}
\usetheme{Rochester}
\setbeamercovered{transparent=25}

\title{Beamer 보기}
\subtitle{Overlay 방법들}
\author[김성빈]{\texttt{chengbinjin@inha.edu}}
\date[\today]{Deep CT to MR Synthesis using Paired and Unpaired Data}
\institute{INHA University}

\begin{document}
\begin{frame}
\titlepage
\end{frame}

\begin{frame}[fragile]
이젠 verbatim 사용 가능
\end{frame}

\begin{frame}[fragile]
\only<beamer>{프리젵네이션에만 보임.}
\end{frame}

\begin{frame}
\frametitle{필요한 가정}
\begin{block}<1->{Answered Question}
\begin{enumerate}
\item<2-> 세 내각의 합은 180이다.
\item<3-> $n$ 각형의 내각의 합은 $180(n-2)$이다.
\end{enumerate}
\end{block}
\begin{block}<4->{Open Question}
삼각형의 세 변의 길이를 $a$, $b$, $c$라고 할 때 $a^2, b^2$ 및 $c^2$의 관계
\end{block}
\end{frame}

\begin{frame}
\frametitle{삼각형의 성질}
\begin{block}<1->{삼각형의 성질 1}
\alert{삼각형}은 한 변의 길이가 나머지 두 변의 길이의 합보다 작다.
\end{block}

\begin{block}<2->{삼각형의 종류}
\begin{itemize}
\item<2-> 예각 삼각형
\item<3-> 직각 삼각형
\item<4-> 둔각 삼각형
\end{itemize}
\end{block}
\end{frame}

\setbeamertemplate{blocks}[rounded][shadow=true]

\begin{frame}
\frametitle{삼각형의 성질}
\begin{block}<1->{삼각형의 성질 1}
\alert{삼각형}은 한 변의 길이가 나머지 두 변의 길이의 합보다 작다.
\end{block}

\begin{block}<2->{삼각형의 종류}
\begin{itemize}
\item<2-> 예각 삼각형
\item<3-> 직각 삼각형
\item<4-> 둔각 삼각형
\end{itemize}
\end{block}
\end{frame}

\begin{frame}
\frametitle{opaqueness와 setbeamercovered}
\setbeamercovered{still covered={\opaqueness<1->{15}},
again covered={\opaqueness<1->{45}}}
\begin{itemize}
\item<1> $\int_0^1 f(x) dx$
\item<2> $S = \sum_{i=1}^n a_i$
\item<3> 세 번째 항목
\item<4> The last item
\end{itemize}
\end{frame}

\end{document}