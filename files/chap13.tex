\documentclass[xcolor=table]{beamer}

\usepackage{multimedia}
\usepackage{kotex}

\usetheme{Madrid}

\title{Beamer 보기}
\subtitle{Overlay 방법들}
\author[김성빈]{\texttt{chengbinjin@inha.edu}}
\date[\today]{Deep CT to MR Synthesis using Paired and Unpaired Data}
\institute{INHA University}

\newcommand{\blueonly}{\only{\color{blue}}}

\begin{document}
\begin{frame}
\titlepage
\end{frame}

\begin{frame}[fragile]
이젠 verbatim 사용 가능
\end{frame}

\begin{frame}[fragile]
\only<beamer>{프리젵네이션에만 보임.}
\end{frame}

\setbeamercovered{transparent=20}
\begin{frame}
\frametitle{pause 보기}
The first slide \\
\pause
두 번쨰 화면 \\ 
\pause
The third slide \\
\pause[2]
두 번쨰 이후에는 계속 보입니다. \\
\pause
The last slide \\
\end{frame}

\begin{frame}
\frametitle{onslide 보기-공간 차지}
\onslide<1>{첫 번쨰 슬라이드에만 보입니다. \\}
\onslide<2>{두 번째 슬라이드에만 보입니다. \\}
\onslide<3>{세 번째 슬라이드 이후 보입니다. \\}
\onslide<4>{네 번쨰 슬라이드에만 보입니다.}
\end{frame}

\begin{frame}
\frametitle{only 보기-공간 없음}
\only<1>{첫 번째 슬라이드에만 보입니다. \\}
\only<2>{두 번째 슬라이드에만 보입니다. \\}
\only<3>{세 번째 슬라이드 이후 보입니다. \\}
\only<4>{네 번째 슬라이드에만 보입니다.}
\end{frame}

\begin{frame}
\frametitle{\texttt{alt} 보기-공간 없음}
\begin{itemize}
\item<1>{첫 번째 슬라이드에만 보입니다. \newline}
\item<2> \alt<2>{두 번째 슬라이드에만 보입니다. \newline}
{네 번째 슬라이드에만 보입니다. \newline}
\item<3->{세 번째 슬라이드 이후 보입니다. \newline}
\item<4>{네 번째 슬라이드에만 보입니다.}
\end{itemize}
\end{frame}



\end{document}